These examples all serve as templates, which can be duplicated to form the start of your project.

There are three examples you should check out first\+:


\begin{DoxyItemize}
\item {\bfseries{\mbox{\hyperlink{class_i_plug_effect}{I\+Plug\+Effect}}}} \+: A basic audio effect which is a volume control.
\item {\bfseries{\mbox{\hyperlink{class_i_plug_instrument}{I\+Plug\+Instrument}}}} \+: An M\+P\+E-\/capable polyphonic synthesiser.
\item {\bfseries{\mbox{\hyperlink{class_i_plug_controls}{I\+Plug\+Controls}}}} \+: A demonstration of the widgets available in the I\+Controls library.
\end{DoxyItemize}

The following examples are more specialized\+:


\begin{DoxyItemize}
\item {\bfseries{\mbox{\hyperlink{class_i_plug_chunks}{I\+Plug\+Chunks}}}} \+: Shows how to store data other than just parameter values in the plug-\/in state and how to get tempo info from the host.
\item {\bfseries{\mbox{\hyperlink{class_i_plug_midi_effect}{I\+Plug\+Midi\+Effect}}}} \+: A basic M\+I\+DI effect plugin. Note\+: only Audio\+Units really have a notion of a midi effect.
\item {\bfseries{\mbox{\hyperlink{class_i_plug_side_chain}{I\+Plug\+Side\+Chain}}}} \+: Demonstrates how to do a plug-\/in with two input buses for effects that require sidechain inputs such as compressors/gates.
\item {\bfseries{\mbox{\hyperlink{class_i_plug_surround_effect}{I\+Plug\+Surround\+Effect}}}} \+: A multichannel volume control effect plug-\/in that should work on different surround buses.
\item {\bfseries{\mbox{\hyperlink{class_i_plug_drum_synth}{I\+Plug\+Drum\+Synth}}}} \+: A drum synthesiser example with multiple output buses.
\item {\bfseries{\mbox{\hyperlink{class_i_plug_responsive_u_i}{I\+Plug\+Responsive\+UI}}}} \+: An example of how to make a responsive \mbox{\hyperlink{class_u_i}{UI}} that adapts to the platform window that can be maximized and resized using the OS window chrome.
\item {\bfseries{\mbox{\hyperlink{class_i_plug_faust_d_s_p}{I\+Plug\+Faust\+D\+SP}}}} \+: A plug-\/in that uses F\+A\+U\+ST to implement its \mbox{\hyperlink{class_d_s_p}{D\+SP}} and J\+I\+T-\/compile F\+A\+U\+ST code in debug builds.
\item {\bfseries{\mbox{\hyperlink{class_i_plug_o_s_c_editor}{I\+Plug\+O\+S\+C\+Editor}}}} \+: Demonstrates how to use the Open Sound \mbox{\hyperlink{struct_control}{Control}} classes in i\+Plug2, as well as the \mbox{\hyperlink{class_i_web_view_control}{I\+Web\+View\+Control}}
\item {\bfseries{\mbox{\hyperlink{class_i_plug_reaper_extension}{I\+Plug\+Reaper\+Extension}}}} \+: This is a template project for making a \href{http://reaper.fm/sdk/plugin/plugin.php}{\texttt{ Reaper Extension}}. No realtime audio processing code, obviously. Making a reaper extension can be painful since it is all based around the Win32 A\+P\+Is. This abstracts away some of the nastyness.
\item {\bfseries{\mbox{\hyperlink{class_i_plug_convo_engine}{I\+Plug\+Convo\+Engine}}}} \+: U\+I-\/less example of W\+D\+L\+\_\+\+Convo\+Engine that reports a delay to the host for plugin-\/delay-\/compensation (P\+DC)
\item {\bfseries{\mbox{\hyperlink{class_i_plug_cocoa_u_i}{I\+Plug\+Cocoa\+UI}}}} \+: An i\+O\+S/mac\+OS project using App\+Kit/\+U\+I\+Kit for the user interface
\item {\bfseries{\mbox{\hyperlink{class_i_plug_swift_u_i}{I\+Plug\+Swift\+UI}}}} \+: An i\+O\+S/mac\+OS only project using Swift\+UI for the user interface
\item {\bfseries{\mbox{\hyperlink{class_i_plug_web_u_i}{I\+Plug\+Web\+UI}}}} \+: An example showing how \mbox{\hyperlink{class_u_i}{UI}} can written in H\+T\+M\+L/\+C\+S\+S/\+JS, using a platform web view 
\end{DoxyItemize}